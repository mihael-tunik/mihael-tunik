\documentclass[12pt,a4paper]{moderncv}
\usepackage{savesym}
\savesymbol{fax}
\usepackage{marvosym}
\restoresymbol{MARV}{fax}
\moderncvstyle{classic}
\moderncvcolor{blue}

\usepackage[english]{babel}
\usepackage[T1]{fontenc}
\usepackage{color}
\usepackage[scale=0.87]{geometry}

\setlength{\hintscolumnwidth}{3.5cm}
\renewcommand*{\namefont}{\fontsize{26}{28}\mdseries\upshape}

\firstname{Михаил}
\familyname{Юрьевич Туник}
\email{mihael.8112@yahoo.com}
\extrainfo{https://github.com/mihael-tunik/}
\photo[100pt]{m_tunik.jpg}
\begin{document}
\makecvtitle

\section{Образование}
\cventry{2013~--- 2017}{бакалавр}{Санкт-Петербург}{Санкт-Петербургский политехнический университет Петра Великого}{кафедра прикладной математики и механики}{}
\cventry{2017~--- 2019}{магистр}{Санкт-Петербург}{Санкт-Петербургский политехнический университет Петра Великого}{кафедра прикладной математики и механики}{}

\subsection{Магистерская работа}
\cventry{2019}{Исследование ядерной оценки плотности вероятности в условиях малой выборки}{}{}{}{}

Работа посвящена исследованию теоретической точности статистической ядерной оценки специального типа в случае конечного размера выборки.

\section{Опыт: 3 года 10 месяцев}
\cventry{август 2019~--- настоящее время}{СПбГУ, Лаборатория им. П.Л.Чебышева}{инженер-исследователь}{}{}{}
\begin{itemize}
\item Принимал участие в разработке ПО для анализа геоданных и сейсмической инверсии. There we used various Gaussian process based regression models (for example, multi-output GP or sparse GP). Also I was involved in research for relevant scientific articles in given subject area. \newline
\item Development of software for solving Riemann problems, which appear in porous media hydrodynamics.
There, for example, I automated certain research pipelines. Also during this period I've developed auxiliary library on C++ for using in Python project via Ctypes. \newline
\item Development MVP for the tool for fine-tuning advanced hydrodynamic simulations in Dumux with Bayesian optimization techniques using botorch. Here I worked with Tensorboard and took part in building custom UI with PyQt5. \newline
\item Last but not least, I take part in local LMS maintaining/management, where my job responsibilities include system administator and devops tasks. \newline
\end{itemize}
\section{Skills}
\cvitem{Programming languages: }{Python, C/C++, Javascript}{}{}{}{}
\cvitem{Mathematical background: }{statistics and probability theory (random functions and fields), linear algebra, calculus.}{}{}{}{}
\cvitem{Computer science background:\hspace*{5mm}}{Standard course of algorithms and numerical methods, various optimization methods, statistical data analisys, ML: regression of all types, table data classification.}{}{}{}{}
\cvitem{More information and keywords: }{
\begin{itemize}
\item General purpose skills: \newline
    \begin{itemize}
        \item Many years of experience with different \textbf{Linux} distributions, system configuration, terminal (bash, Unix commands); \newline
        \item \textbf{Git} VCS, managing repositories in \textbf{Bitbucket} and \textbf{GitHub} (pull-requests, Github Actions CI, reviews), \textbf{Notion} for task-tracking; \newline
        \item Remote access via \textbf{ssh}, familiar with \textbf{Docker} and \textbf{docker-compose}; \newline
        \item Extensive experience with \textbf{Python} toolchain and ecosystem: building up Python packages from scratch with \textbf{setuptools},
managing packages and project installations with \textbf{venv} or \textbf{Anaconda}; \newline
        \item Testing with \textbf{pytest}, \textbf{Sphinx} for automated documentation; \newline
    \end{itemize}
\item Experience as engineer-researcher: \newline
    \begin{itemize}
        \item Work on project sketches in \textbf{Jupyter Notebooks} and \textbf{Google Colab}; \newline
        \item Work skills with \textbf{Pandas} and sklearn, experience with GPFlow, \textbf{GPy} for work with Gaussian Processes, gradient boosting with \textbf{CatBoost};\newline
        \item Advanced work with \textbf{LaTeX} for scientific texts and presentations; \newline
        \item Experience with Matlab/Octave, Mathematica, R; \newline
        \item Basic experience with Tensorflow/Torch frameworks; \newline
    \end{itemize} 
\item Some experience from desktop-dev:\newline
    \begin{itemize}
        \item UI development with \textbf{PyQt5}, Qt Creator IDE, PyInstaller for bulding binaries; \newline
        \item Experience in writing detailed documentation for code and UI, building auto-generated Excel reports with openpyxl; \newline
    \end{itemize}
\item Some experience from web-dev: \newline
    \begin{itemize}
        \item Basic knowledge of PostgreSQL, SQLite (and ORM, in general, with Django); \newline 
        \item Experience with MinIO S3 server deployment; \newline 
        \item Some experience with HTML, CSS/SCSS and Bootstrap; \newline
        \item Basic knowledge of Javascript ecosystem (npm, React.js, gulp, Axios); \newline
        \item Some experience with API building using Django REST Framework or FastAPI.\newline
        \item Basic knowledge of HTTP protocol, Nginx and ngrok.\newline
    \end{itemize}
\item Some experience with C/C++ (parallel computations with OpenMP, make-files and CMake, building small .so libs); \newline \newline
\end{itemize}
}{}{}{}{}

\section{Languages}
\cvitemwithcomment{Russian}{Native speaker}{}{}
\cvitemwithcomment{English}{Upper-Intermediate}{}{}
\section{Online courses}
\cvitemwithcomment{Stepik}{October 2022}{\textbf{Hadoop. Система обработки больших данных.}}{Learned basic things about Hadoop ecosystem, including HDFS, MapReduce, HBase and also Apache Spark. Practicing in writing code in Scala as a bonus.\newline\newline
\textit{Result: certificate with distinction, >90\% score.}}\newline\newline
\cvitemwithcomment{Stepik}{January 2023}{\textbf{Apache Airflow для аналитиков.}}{Learned basic things about Apache Airflow, DAGs and ETL in general.\newline\newline \textit{Result: certificate.}} 
\end{document}
