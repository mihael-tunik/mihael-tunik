\documentclass[12pt,a4paper,sans]{moderncv}
\usepackage{savesym}
\savesymbol{fax}
\usepackage{marvosym}
\restoresymbol{MARV}{fax}
\moderncvstyle{classic}

\definecolor{carmine}{rgb}{0.56, 0.71, 0.78}
\colorlet{color1}{carmine}
%\moderncvcolor{orange}

\usepackage[english]{babel}
\usepackage[T1]{fontenc}
\usepackage{color}
\usepackage[scale=0.87]{geometry}


\setlength{\hintscolumnwidth}{3.5cm}
\renewcommand*{\namefont}{\fontsize{26}{28}\mdseries\upshape}
\renewcommand{\familydefault}{\sfdefault}

\firstname{Mihael}
\familyname{Tunik}
\email{mihael.8112@yahoo.com}
\social[github]{mihael-tunik}
\photo[100pt]{m_tunik.jpg}
\begin{document}
\makecvtitle

\section{About me}
Programmer with versatile experience in IT and computer science.
I do believe that modern scientific research process requires significant programming skills (and ready to provide them).

\section{Education}
\cventry{2013~--- 2017}{Bachelor degree}{Saint-Petersburg}{Peter the Great St.Petersburg Polytechnic University}{departament of applied mathematics and mechanics}{}
\cventry{2017~--- 2019}{Master degree}{Saint-Petersburg}{Peter the Great St.Petersburg Polytechnic University}{departament of applied mathematics and mechanics}{}

\subsection{Master thesis}
\cventry{2019}{Special kernel density estimator for finite sample size conditions}{}{}{}{}
Work was dedicated to research of theoretical accuracy of statistical kernel density estimator of special type for finite sample size conditions.

\section{Experience: >5 years}
\cventry{august 2019~--- now}{Saint-Petersburg State University, Chebyshev Laboratory}{engineer-researcher}{}{}{}
\begin{itemize}
\item Here, I started as an intern in the small team, where we're developing statistical instruments for geo-data analysis and seismic inversion. 
There we extensively used various Gaussian process based regression models and various techniques for data-processing. \newline \newline
Typical tasks: 
 \begin{itemize}
  \item research for relevant scientific articles;
  \item automate research pipeline;
  \item integrate and test new submodule in codebase; \newline
 \end{itemize}
\item Then, I continued to work as engineer-researcher on development the tool for fine-tuning advanced 
hydrodynamic simulations in Dumux with Bayesian optimization techniques.  
Among other things as a researcher I took part in implementing experimental software for solving Riemann problems. \newline \newline
Typical tasks: 
 \begin{itemize}
  \item reorganize project codebase, fix architecture issues;
  \item rewrite algorithmic core for optimization; \newline
 \end{itemize}
\item Latest project, where I work mostly with ML-pipelines for classification/recognition timeseries data from sensors of gas-analyzer.
Developed window-based method for timeseries classification based on classic and gradient boosting models.
\newline \newline
Typical tasks: 
 \begin{itemize}
  \item Explore the data and develop strategies for handling it;
  \item Develop project research pipeline completely from scratch;
  \item Propose and develop different models for solving stated ML-problems; \newline
 \end{itemize}
\item Actually during my work I've created even more things: like microservices for convenient remote access to advanced simulator software or custom desktop UI for one of ours subprojects. \newline
\end{itemize}
\section{Technical skills}
\hspace{10mm} Started my research career in fields of statistics and probability theory. Also I'm competent enough in numerical methods and algorithms. \newline
In the recent projects I had a lot of practice with \textbf{statistical data analisys} and \textbf{ML} (hypothesis testing, feature engineering, timeseries data classification). \newline
 \begin{itemize}
 \item General purpose skills: \newline
    \begin{itemize}
        \item Extensive experience with \textbf{Python} toolchain and ecosystem: building up Python packages from scratch with \textbf{setuptools},
managing things with \textbf{venv} or \textbf{Anaconda}; \newline
        \item Many years of experience with different \textbf{Linux} distributions (Ubuntu, Fedora, Mint), system configuration (bash, Unix commands); \newline
        \item Proficient with \textbf{Git}, managing repositories, \textbf{Github Actions} CI; \textbf{Notion} for task-tracking; \newline
        \item Familiar with \textbf{Docker} and \textbf{docker-compose}; \newline
        \item Familiar with testing (\textbf{pytest}), profiling and automated documentation tools; \newline
    \end{itemize}    
 \item Experience as engineer-researcher:\newline
    \begin{itemize}
        \item Proficient with \textbf{numpy, scipy, sklearn}; familiar with \textbf{Pandas and Polars} dataframe engines;
        \item Decision trees and gradient boosting with \textbf{CatBoost/LightGBM/XGBoost}; 
        \item Worked with ensembles and various \textbf{model stacking} techniques, \textbf{multi-staged classifiers}; \newline
        \item Familiar with Tensorflow and Keras; \newline
        \item Advanced \textbf{LaTeX} for scientific texts and presentations; \newline
    \end{itemize} 
 \item Some experience from desktop-dev:\newline
    \begin{itemize}
        \item UI development with \textbf{PyQt5}, Qt Creator IDE, PyInstaller for bulding binaries; \newline
        \item Experience in writing detailed documentation for code and UI; \newline
    \end{itemize}
 \item Some experience from web-dev: \newline
    \begin{itemize}
        \item Some experience from backend: 
          HTTP protocol, Nginx, \textbf{Flask}, Django, testing APIs with \textbf{Postman};\newline
        \item Basic experience with databases (PostgreSQL, ClickHouse, SQLite) and key-value stores;\newline
        \item Some experience from frontend: HTML, CSS/SCSS, static site generators; \newline
    \end{itemize}
 \item Some experience with C/C++ (OpenMP, CMake, Valgrind and building small .so libs), \textbf{Python C API} and \textbf{Ctypes}, worked with low-level C API for XGBoost and Eigen libs; \newline \newline
 \end{itemize} 
\section{Languages}
\cvitemwithcomment{Russian}{C2, Native speaker}{}{}
\cvitemwithcomment{English}{B2, Upper-Intermediate}{}{}
\cvitemwithcomment{German }{A2, Beginner}{}{}
\section{Articles and preprints}
 \begin{itemize}
   \item Classification of Graphene-Based Electronic Nose Measurements with Gradient-Boosted Decision Trees. Available at SSRN: https://ssrn.com/abstract=5041771
 \end{itemize} 
\section{Personal webpage}
https://mihael-tunik.github.io/ \newline \newline
Here I write small articles about programming and computer science and make experiments with static site generators.
\end{document}
