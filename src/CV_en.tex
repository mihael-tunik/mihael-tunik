\documentclass[12pt,a4paper]{moderncv}
\usepackage{savesym}
\savesymbol{fax}
\usepackage{marvosym}
\restoresymbol{MARV}{fax}
\moderncvstyle{classic}
\moderncvcolor{purple}

\usepackage[english]{babel}
\usepackage[T1]{fontenc}
\usepackage{color}
\usepackage[scale=0.87]{geometry}

\setlength{\hintscolumnwidth}{3.5cm}
\renewcommand*{\namefont}{\fontsize{26}{28}\mdseries\upshape}
\renewcommand{\familydefault}{\sfdefault}

\firstname{Mihael}
\familyname{Tunik}
\email{mihael.8112@yahoo.com}
\extrainfo{https://github.com/mihael-tunik/}
\photo[100pt]{m_tunik.jpg}
\begin{document}
\makecvtitle

\section{About me}
Programmer with versatile experience in IT and computer science.
I do believe that modern scientific research process requires significant programming skills (and ready to provide them).
Seeking a position as a developer-researcher to enhance my career growth.

\section{Education}
\cventry{2013~--- 2017}{Bachelor degree}{Saint-Petersburg}{Peter the Great St.Petersburg Polytechnic University}{departament of applied mathematics and mechanics}{}
\cventry{2017~--- 2019}{Master degree}{Saint-Petersburg}{Peter the Great St.Petersburg Polytechnic University}{departament of applied mathematics and mechanics}{}

\subsection{Master thesis}
\cventry{2019}{Special kernel density estimator for finite sample size conditions}{}{}{}{}
Work is dedicated to research of theoretical accuracy of statistical kernel density estimator of special type for finite sample size conditions.

\section{Experience: 4 years and 1 month}
\cventry{august 2019~--- now}{Saint-Petersburg State University, Chebyshev Laboratory}{engineer-researcher}{}{}{}
\begin{itemize}
\item Team work on developement of statistical instruments for geo-data analysis and seismic inversion. There we extensively used various Gaussian process based regression models and various technics for data-processing. Also I helped with research for relevant scientific articles and automated several research pipelines. \newline

\item Development MVP for the tool for fine-tuning advanced hydrodynamic simulations in Dumux with Bayesian optimization techniques. Here I also was responsible for building custom UI.
Among other things as a researcher I took part in implementing experimental software for solving Riemann problems. \newline

\item During my work I've created couple microservices for convenient remote access to advanced simulator software. \newline

\end{itemize}
\section{Skills}
\cvitem{Programming languages: }{Python [advanced], C/C++ [medium], SQL, R and Javascript [basic];}{}{}{}{}
\cvitem{Mathematical background: }{Mathematical statistics and probability theory, linear algebra, calculus.}{}{}{}{}
\cvitem{Computer science background:\hspace*{5mm}}{Algorithms and numerical methods, 
statistical data analisys: LDA, PCA, hypothesis testing, ML: regression, table data classification.}{}{}{}{}
\cvitem{More information and keywords: }{
 \begin{itemize}
 \item General purpose skills: \newline
    \begin{itemize}
        \item Many years of experience with different \textbf{Linux} distributions, system configuration, terminal (bash, Unix commands); \newline
        \item Proficient with \textbf{Git}, managing repositories: pull-requests, Github Actions CI, reviews; \textbf{Notion} for task-tracking; \newline
        \item Remote access via \textbf{ssh}, familiar with \textbf{Docker} and \textbf{docker-compose}; \newline
        \item Extensive experience with \textbf{Python} toolchain and ecosystem: building up Python packages from scratch with \textbf{setuptools},
managing packages and project installations with \textbf{venv} or \textbf{Anaconda}; \newline
       % \item Testing with \textbf{pytest}, profiling with \textbf{cProfile} and \textbf{Sphinx} for automated documentation; \newline
    \end{itemize}    
 \item Experience as engineer-researcher:\newline
    \begin{itemize}
        \item Work on project sketches in \textbf{Jupyter Notebooks} and \textbf{Google Colab}; \newline
        \item Proficient with \textbf{numpy, pandas, sklearn}; GPFlow, GPy for work with GP models; botorch, bayes\_opt for Bayesian optimization, 
        gradient boosting with \textbf{CatBoost}; experience with Torch framework; \newline
        \item Advanced work with \textbf{LaTeX} for scientific texts and presentations; \newline
    \end{itemize} 
 \item Some experience from desktop-dev:\newline
    \begin{itemize}
        \item UI development with \textbf{PyQt5}, Qt Creator IDE, PyInstaller for bulding binaries; \newline
        \item Experience in writing detailed documentation for code and UI; \newline
    \end{itemize}
 \item Some experience from web-dev: \newline
    \begin{itemize}
        \item Some experience from backend, sufficient to develop inelaborate microservice from scratch: 
          HTTP protocol, Nginx, \textbf{Flask}, Django ecosystem (ORM, REST Framework), testing APIs with \textbf{Postman};\newline
        \item Some experience with databases (PostgreSQL, SQLite) and key-value stores: Redis and memcached;\newline
        \item Some experience from frontend: 
          HTML, CSS/SCSS, basic knowledge of Javascript ecosystem (npm, React.js, axios); \newline
    \end{itemize}
 \item Some experience with C/C++ (parallel computations with OpenMP, make-files and CMake, building small .so libs), linking C++ and Python via \textbf{Ctypes}; \newline \newline
\end{itemize}
}{}{}{}{}
\section{Languages}
\cvitemwithcomment{Russian}{Native speaker}{}{}
\cvitemwithcomment{English}{Upper-Intermediate}{}{}
\section{Online courses}
\cvitemwithcomment{Stepik}{October 2022}{\textbf{Hadoop. System for big data processing.}}{Learned basic things about Hadoop ecosystem, including HDFS and MapReduce. \newline\newline
\textit{Result: certificate with distinction, >90\% score.}}\newline\newline
\cvitemwithcomment{Stepik}{January 2023}{\textbf{Apache Airflow for analysts.}}{Learned basic things about Apache Airflow, DAGs and ETL in general.\newline\newline \textit{Result: certificate.}} 
\end{document}
